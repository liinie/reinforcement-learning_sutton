\chapter{蒙特卡罗方法}
这一章是我们本书探讨的第一个学习方法来评估价值函数和发现最优策略。
不像以往的章节,这里我们不会假设一个完全已知的环境。
蒙特卡罗方法仅需\textit{经验}---从实际或模拟的与环境的互动中获得的状态样本序列,动作和奖励。
从\textit{实际经验}中学习是很值得关注的,因为它不需要环境动态的先验知识,却仍然能够做出最优决策。
从\textit{模拟的}经验中学习也是很强大的。
尽管我们需要模型,这个模型也只需要生成样本转移(sample transitions),而不是所有可能转移的完整概率分布,像动态分布(DP)要求的那样。
令人惊奇的是,在许多情况下,按照所需概率分布生成经验样本是简单的,而得到精确的分布却不可行。

[Sachen kommen noch!]


\section{蒙特卡罗预测}
[Sachen kommen hier noch!]
\paragraph{Example 5.1: Blackjack} 
二十四点纸牌赌博游戏的目的是得到尽可能大的点数,但这个点数不能超过21。
所有人头牌(J, Q, K)算作10,A可以算作1或11。
我们这里使用的版本是每一个玩家独立与庄家(dealer)对抗。
游戏开始时,玩家与庄家各得两张牌。
发牌者的一张牌面朝上为“明牌”,其他牌点数朝下为“暗牌”。
如果玩家这时已经达到21点(一个A和一个10点), 这个牌面叫\textit{自然(natural)}。
玩家赢得胜利,除非庄家也是自然,那么游戏平局(draw)。
如果玩家没有自然,他要继续一个一个地叫牌(\textit{hits}),直到他停牌(\textit{sticks}),或者超过21点就是爆牌(\textit{goes bust})。
如果玩家爆牌,玩家输;
如果他停牌,就轮到庄家叫牌。
庄家叫牌或停牌的策略是固定的:
大于等于17点时庄家停牌,没到则叫牌。
如果庄家爆牌,则玩家胜;
如果庄家没有爆牌,则其余玩家要揭开手中的牌,比较点数,点数大的取胜。

[Sachen kommen hier noch!]

\paragraph{Example 5.2: Soap Bubble}
试想一个线框围成的闭环浸在肥皂水里形成的一个肥皂膜或肥皂泡。
如果线框的几何形状是已知的不规则的形状,怎样才能计算出肥皂泡表面的形状?
这个表面的特性是每一点受到周围点的合力为零(否则形状会改变)。
这表示表面上任一点的高度是围绕在以这一点为中心的小圆中所有点的高度的平均值。
另外,表面的边界必须与线框相接。
通常解决这类问题的方法是用网格覆盖表面,并通过迭代计算解出网格点上表面的高度。 
表面边界的网格点被规定落在线框上,所有其他的点被调整为与它相邻四点高度的平均值。
这个过程会进行迭代,就像动态规划的迭代策略评估,并最终收敛于一个近似的表面。

蒙特卡罗方法最初的设计目的就是用于解决类似的问题。

